%% LyX 2.2.1 created this file.  For more info, see http://www.lyx.org/.
%% Do not edit unless you really know what you are doing.
\documentclass[11pt,english]{article}
\usepackage[T1]{fontenc}
\usepackage[latin9]{inputenc}
\usepackage{geometry}
\geometry{verbose,tmargin=1.25in,bmargin=1.25in,lmargin=1.25in,rmargin=1.25in}
\usepackage{url}
\usepackage{amsmath}
\usepackage{amsthm}
\usepackage{amssymb}
\usepackage{graphicx}
\usepackage{setspace}
\usepackage{wasysym}
\usepackage[authoryear]{natbib}
\onehalfspacing

\makeatletter
%%%%%%%%%%%%%%%%%%%%%%%%%%%%%% Textclass specific LaTeX commands.
  \theoremstyle{plain}
  \newtheorem{prop}{\protect\propositionname}
  \theoremstyle{plain}
  \newtheorem{lem}{\protect\lemmaname}

%%%%%%%%%%%%%%%%%%%%%%%%%%%%%% User specified LaTeX commands.
\usepackage{babel}


% set fonts for nicer pdf view
\IfFileExists{lmodern.sty}{\usepackage{lmodern}}{}



\usepackage{babel}
\providecommand{\lemmaname}{Lemma}
  \providecommand{\propositionname}{Proposition}

\date{December 2016 draft}



\usepackage{babel}
\providecommand{\lemmaname}{Lemma}
  \providecommand{\propositionname}{Proposition}
\usepackage{varioref}

\@ifundefined{showcaptionsetup}{}{%
 \PassOptionsToPackage{caption=false}{subfig}}
\usepackage{subfig}
\makeatother

\usepackage{babel}
  \providecommand{\lemmaname}{Lemma}
  \providecommand{\propositionname}{Proposition}

\begin{document}

\title{Online Appendix for \\Breakable commitments: \\
 present-bias, client protection and bank ownership forms}

\author{Karna Basu \& Jonathan Connin

\appendix
%dummy comment inserted by tex2lyx to ensure that this paragraph is not empty

\section{Online (longer) Appendix: CRRA Derivations and Proofs}

\subsection{Full-Commitment}

For the monopolist bank that offers full-commitment, the solution
is determined by the first-order condition and the consumer's participation
constraint: 
\begin{align}
C_{0}^{mF} & =\left(1,\beta^{\frac{1}{\rho}},\beta^{\frac{1}{\rho}}\right)\cdot\left(\frac{U_{0}^{A}\left(1-\rho\right)}{1+2\beta^{\frac{1}{\rho}}}\right)^{\frac{1}{1-\rho}}\label{eq:c-mf}\\
\Pi_{0}\left(C_{0}^{mF};Y_{0}\right) & =y-\left(U_{0}^{A}\left(1-\rho\right)\right)^{\frac{1}{1-\rho}}\left(1+2\beta^{\frac{1}{\rho}}\right)^{\frac{-\rho}{1-\rho}}\label{eq:pi-mf}
\end{align}

For the competitive banks that offer full-commitment, the solutions
is determined by the first-order condition and the bank's participation
constraint: 
\begin{equation}
C_{0}^{F}=\left(1,\beta^{\frac{1}{\rho}},\beta^{\frac{1}{\rho}}\right)\cdot\left(\frac{y}{1+2\beta^{\frac{1}{\rho}}}\right)\label{eq:c-cf}
\end{equation}


\subsection{Renegotiation}

Given an existing contract $C_{1}$, a monopolist bank that renegotiates
in period 1 will offer the following new contract: 
\begin{equation}
C_{1}^{m1}\left(C_{1}\right)=\left(1,\beta^{\frac{1}{\rho}}\right)\cdot\left(\frac{c_{1}^{1-\rho}+\beta c_{2}^{1-\rho}}{1+\beta^{\frac{1}{\rho}}}\right)^{\frac{1}{1-\rho}}\label{eq:c-r}
\end{equation}

The corresponding profit gains from renegotiation are: 
\begin{equation}
\Pi_{1}\left(C_{1}^{m1}\left(C_{1}\right);C_{1}\right)=\left(c_{1}+c_{2}\right)-\left(c_{1}^{1-\rho}+\beta c_{2}^{1-\rho}\right)^{\frac{1}{1-\rho}}\left(1+\beta^{\frac{1}{\rho}}\right)^{\frac{-\rho}{1-\rho}}\label{eq:pi-r}
\end{equation}

An alternate way to restate the above is the following: Let $s$ and
$\alpha$ be defined such that $c_{1}=\alpha s$ and $c_{2}=\left(1-\alpha\right)s$.
Then: 
\begin{equation}
C_{1}^{m1}\left(C_{1}\right)=\left(s,s\beta^{\frac{1}{\rho}}\right)\cdot\left(\frac{\alpha^{1-\rho}+\beta\left(1-\alpha\right)^{1-\rho}}{1+\beta^{\frac{1}{\rho}}}\right)^{\frac{1}{1-\rho}}\label{eq:c-r-alpha}
\end{equation}

Profit gains from renegotiation become: 
\begin{equation}
\Pi_{1}\left(C_{1}^{m1}\left(C_{1}\right);C_{1}\right)=\left(s\right)\left(1-\left(\alpha^{1-\rho}+\beta\left(1-\alpha\right)^{1-\rho}\right)^{\frac{1}{1-\rho}}\left(1+\beta^{\frac{1}{\rho}}\right)^{\frac{-\rho}{1-\rho}}\right)\label{eq:renegotiation-profits}
\end{equation}

By construction, profits from renegotiation are strictly positive
(and increasing in $s$), except in the special case where $C_{1}$
is optimal from period 1's perspective ($\left(1-\alpha\right)=\beta^{\frac{1}{\rho}}\alpha$),
in which case they are $0$. It can also easily be confirmed that
profits from renegotiation fall in $\alpha$ as long as the allocation
is such that period 1 would like a larger $\alpha$ than the current
contract offers.

\textbf{\textit{ALTERNATE Proof of Proposition 1:}} When we have the full-commitment contract under competition we can write the binding no-renegotiation constraint as: 
\begin{equation}
y-c_0^F-\kappa =(1+\beta^\frac{1}{\rho})c_1^1 \\
\end{equation}
where we have used the fact that $c_0^F+c_1^F+c_2^F=y$ and the fact that at any renegotiated contract $R$ we must have $c_1^2=\beta^\frac{1}{\rho}c_1^1$ and hence $c_1^1+c_2^1 =(1+\beta^\frac{1}{\rho})c_1^1$. So at a point like $R$ in the diagram
\begin{equation*}
c_1^1=\frac{y-c_0^F-\kappa}{1+\beta^\frac{1}{\rho}}$
\end{equation}
At $R$ we also have $u(c_1^F)+\beta u(c_2^F)=u(c_1^1)+\beta u(c_2^1)$ (i.e. on the indifference curve $FR$). Putting this all together and simplifying we can solve to find the $\bar \kappa$ at which the no-renegotiation constraint just binds:
$$u(c_1^F)+\beta u(c_2^F)=u(c_1^1)+\beta u(c_2^1)$$
$$u(\beta^\frac{1}{\rho} c_0^F)+\beta u(\beta^\frac{1}{\rho} c_0^F)=(1+\beta^\frac{1}{\rho})u(c_1^1)$$
$$(1+\beta) \beta^\frac{1-\rho}{\rho} u(c_0^F)
=(1+\beta^\frac{1}{\rho}) \cdot 
u \left(\frac{y-c_0^F-\kappa}{1+\beta^\frac{1}{\rho}} \right)$$
\begin{equation}
\bar \kappa = y -(2+\beta^\frac{1}{\rho} )c_0^F
\end{equation}
If we substitute in our earlier solution for $c_0^F$ this becomes:
\begin{equation}
\bar \kappa = y -)\frac{(2+\beta^\frac{1}{\rho} }{1+2 \beta^\frac{1}{\rho} }
\end{equation}


T
Substituting for $s$ from the competitive full-commitment contract,
we can rewrite the above condition as: 
\begin{equation}
k\geq\frac{2\beta^{\frac{1}{\rho}}y}{1+2\beta^{\frac{1}{\rho}}}\left(1-\frac{1}{2}\left(1-\beta\right)^{\frac{1}{1-\rho}}\left(1+\beta^{\frac{1}{\rho}}\right)^{\frac{-\rho}{1-\rho}}\right)\equiv\bar{\kappa}\label{eq:max-kappa}
\end{equation}
Since in any full-commitment contract (monopoly and competition),
$s$ will be no larger than $\frac{2\beta^{\frac{1}{\rho}}y}{1+2\beta^{\frac{1}{\rho}}}$,
no full-commitment contract will be renegotiated if $\kappa\geq\bar{\kappa}$.

(c) If $\kappa<\bar{\kappa}$, condition \ref{eq:max-kappa} fails,
so the competitive full-commitment contract cannot survive.

(b) The monopolist full-commitment contract will not survive if $s$
is sufficiently large. Since $s^{mF}$ is exponentially increasing
in $U_{0}^{A}$, there must be some $U^{m}$ such that the contract
will not survive if and only if $U_{0}^{A}>U^{m}$. Since the contract
cannot survive at $U_{0}^{F}$ (here, the contract is identical to
the competitive contract) , $U^{m}<U_{0}^{F}$. $\Square$

\subsection{Renegotiation-Proof Contracts}

\subsubsection{Sophisticated Hyperbolic Discounters}

When the renegotiation-proofness constraint binds, consumption in
periods 1 and 2 must satisfy: 
\begin{equation}
\left(s\right)\left(1-\left(\alpha^{1-\rho}-\beta\left(1-\alpha\right)^{1-\rho}\right)^{\frac{1}{1-\rho}}\left(1+\beta^{\frac{1}{\rho}}\right)^{\frac{-\rho}{1-\rho}}\right)=\kappa\label{eq:rp-constraint}
\end{equation}

For any $s$, there may be two values of $\alpha$ that satisfy the
constraint with equality\textendash one with too little consumption
relative to period 1's optimal, one with too much consumption relative
to period 1's optimal. The relevant value for us is the first. This
defines a continuous function $\alpha\left(s\right)$. 
\begin{equation}
\alpha\left(s\right)=min\left\{ \alpha:\left(s\right)\left(1-\left(\alpha^{1-\rho}+\beta\left(1-\alpha\right)^{1-\rho}\right)^{\frac{1}{1-\rho}}\left(1+\beta^{\frac{1}{\rho}}\right)^{\frac{-\rho}{1-\rho}}\right)=\kappa\right\} \label{eq:alpha}
\end{equation}

As $s$ rises, to continue satisfying the constraint we must have
$\alpha\left(s\right)$ rising too (if fractions stayed constant,
profits from renegotiation would rise).

We can also rewrite the first-order condition of the bank's maximization
problem using the new notation. For any $s$ and $\alpha$, let $V\left(s,\alpha\right)=u\left(\alpha s\right)+u\left(\left(1-\alpha\right)s\right)$.
This is the discounted utility over periods 1 and 2, from period 0's
perspective. The solution, $C_{0}=$$\left(c_{0},\alpha s,\left(1-\alpha\right)s\right)$,
must satisfy: 
\begin{equation}
\frac{du\left(c_{0}\right)}{dc_{0}}=\beta\frac{dV\left(s,\alpha\right)}{ds}\label{eq:du-dv}
\end{equation}

In other words, at the profit-maximizing contract the marginal dollar
should be equally valuable whether consumed immediately or distributed
across future periods.

\textbf{\textit{Proof of Proposition 2:}} (a) Since the full-commitment
profit-maximizing contract was uniquely determined, and since it does
not satisfy the renegotiation-proofness constraint, the renegotiation-proof
contract must yield lower profits than the full-commitment contract
does.

(b) Clearly, $\Pi_{0}\left(C_{0}^{mP};Y_{0}\right)$ falls strictly
in $U_{0}^{A}$ (if autarky utility falls, the bank can always do
better, at least by simply lowering $c_{0}$). Since at $U_{0}^{A}=U^{m}$,
$\Pi_{0}\left(C_{0}^{mP};Y_{0}\right)=\Pi_{0}\left(C_{0}^{mF};Y_{0}\right)>0$
and at $U_{0}^{A}=U_{0}^{F}$, $\Pi_{0}\left(C_{0}^{mP};Y_{0}\right)<\Pi_{0}\left(C_{0}^{mF};Y_{0}\right)=0$,
there must be some intermediate autarky utility above which the bank's
maximized profits will be negative.

(c) consider any $c_{0}\leq c_{0}^{mF}$ and $s$ such that $U_{0}\left(c_{0},\frac{s}{2},\frac{s}{2}\right)=U_{0}^{A}$.
We can find the corresponding $\bar{s}$ that, while satisfying the
participation constraint, gives the same utility from period 0's perspective:

\begin{align}
V\left(s,\frac{1}{2}\right) & =\bar{V}\left(\bar{s},\alpha\left(\bar{s}\right)\right)\label{eq:v=00003D00003Dvbar1}\\
\Rightarrow2\frac{\left(\frac{1}{2}s\right)^{1-\rho}}{1-\rho} & =\frac{\left(\alpha\left(\bar{s}\right)\bar{s}\right)^{1-\rho}}{1-\rho}+\frac{\left(\left(1-\alpha\left(\bar{s}\right)\right)\bar{s}\right)^{1-\rho}}{1-\rho}\label{eq:v-vbar2}\\
\Rightarrow\bar{s} & =s\left(\frac{2\left(\frac{1}{2}\right)^{1-\rho}}{\alpha\left(\bar{s}\right)^{1-\rho}+\left(1-\alpha\left(\bar{s}\right)\right)^{1-\rho}}\right)^{\frac{1}{1-\rho}}\label{eq:v-vbar3}
\end{align}

From this, we get the following inequality: 
\begin{align}
\frac{dV\left(\bar{s},\alpha\left(\bar{s}\right)\right)}{ds} & =\bar{s}^{-\rho}\left(\alpha\left(\bar{s}\right)^{1-\rho}+\left(1-\alpha\left(s\right)\right)^{1-\rho}\right)+\frac{d\alpha\left(\bar{s}\right)}{ds}\bar{s}^{1-\rho}\left(\alpha\left(\bar{s}\right)^{-\rho}-\left(1-\alpha\right)^{-\rho}\right)\label{eq:dv-ds1}\\
 & <\bar{s}^{-\rho}\left(\alpha\left(\bar{s}\right)^{1-\rho}+\left(1-\alpha\left(s\right)\right)^{1-\rho}\right)\label{eq:dv-ds2}\\
 & =s^{-\rho}\left(2\left(\frac{1}{2}\right)^{1-\rho}\right)\left(\frac{2\left(\frac{1}{2}\right)^{1-\rho}}{\alpha\left(\bar{s}\right)^{1-\rho}+\left(1-\alpha\left(\bar{s}\right)\right)^{1-\rho}}\right)^{\frac{-1}{1-\rho}}\label{eq:dv-ds3}\\
 & <s^{-\rho}\left(2\left(\frac{1}{2}\right)^{1-\rho}\right)=\frac{dV\left(s,\frac{1}{2}\right)}{ds}\label{eq:dv-ds4}
\end{align}

The first line above splits the effect of $s$ on $V$ into two\textendash the
first term represents the change in utility holding $\alpha$ constant,
and the second term represents the (negative) effect of the further
skewing of consumption that results from a rise in $s$. The final
inequality follows from the fact that, since $\bar{s}>s^{mF}$, the
renegotiation constraint must bind so that $\alpha\left(\bar{s}\right)>\frac{1}{2}$.
Finally, the following inequality holds: 
\begin{align}
\frac{du\left(c_{0}\right)}{dc_{0}} & \geq\frac{du\left(c_{0}^{mF}\right)}{dc_{0}}=\beta\frac{dV\left(s^{mF},\frac{1}{2}\right)}{ds}\label{eq:dudc-dvds1}\\
 & \geq\beta\frac{dV\left(s,\frac{1}{2}\right)}{ds}>\beta\frac{dV\left(\bar{s},\alpha\left(\bar{s}\right)\right)}{ds}\label{eq:dudc-dvds2}
\end{align}

We have shown that at any $c_{0}\leq c_{0}^{mF}$, for a contract
that satisfies the renegotiation-proofness constraint, the marginal
utility of period 0 consumption will be higher than the discounted
marginal utility of future consumption, so the bank could earn strictly
higher profits by raising $c_{0}$ and lowering $s$ further. Therefore,
in the renegotiation-proof contract, $c_{0}^{mP}>c_{0}^{mF}$. $\Square$

\textbf{\emph{Proof of Proposition 3:}} (a) We know that $U_{0}\left(C_{0}^{F}\right)=U_{0}^{F}$.
By assumption, since the renegotiation-proofness constraint is binding,
the renegotiation-proof contract cannot offer the optimal consumption
path. Therefore $U_{0}\left(C_{0}^{P}\right)<U_{0}\left(C_{0}^{F}\right)$.

(b) Consider $\bar{U}$, as constructed in Proposition 2. If $U_{0}^{A}>\bar{U}$,
it is impossible to construct a contract that earns nonnegative profits
and gives the consumer at least autarky utility. Therefore, any contract
that earns zero profits would give the period 0 consumer less than
autarky utility. (As an aside, observe that $\bar{U}=U_{0}\left(C_{0}^{P}\right)$.)

(c) At the full-commitment contract: 
\begin{equation}
\frac{du\left(c_{0}^{F}\right)}{dc}=\beta\frac{dV\left(s^{F},\frac{1}{2}\right)}{ds}=\left(s^{F}\right)^{-\rho}\left(2\left(\frac{1}{2}\right)^{1-\rho}\right)
\end{equation}
Consider a renegotiation-proof contract with $c_{0}=c_{0}^{F}$. To
keep bank profits zero, this contract would also have $s=s^{F}$.
But in the renegotiation-proof contract, $s$ must be divided according
to the fraction $\alpha\left(s^{F}\right)$. So: 
\begin{align}
\frac{dV\left(s^{F},\alpha\left(s^{F}\right)\right)}{ds} & =\left(s^{F}\right)^{-\rho}\left(\alpha\left(s^{F}\right)^{1-\rho}+\left(1-\alpha\left(s^{F}\right)\right)^{1-\rho}\right)\nonumber \\
 & +\frac{d\alpha\left(s^{F}\right)}{ds}\left(s^{F}\right)^{1-\rho}\left(\alpha\left(s^{F}\right)^{-\rho}-\left(1-\alpha\left(s^{F}\right)\right)^{-\rho}\right)\label{eq:dv-ds-comp}
\end{align}
The first term\textendash the direct effect of a change in $s$\textendash is
weakly less than $\frac{dV\left(s^{F},\frac{1}{2}\right)}{ds}$ if
$\rho\leq1$ and strictly greater if $\rho>1$. The second term\textendash the
component of $\frac{dV}{ds}$ that is driven by the change in $\alpha$\textendash is
strictly negative. Therefore, if $\rho<1$, $\frac{dV\left(s^{F},\alpha\left(s^{F}\right)\right)}{ds}<\frac{dV\left(s^{F},\frac{1}{2}\right)}{ds}=\frac{du\left(c_{0}^{F}\right)}{dc}$,
so the renegotiation-proof contract must satisfy $c_{0}^{P}>c_{0}^{F}$.

Next, we consider the case when $\rho>1$. We can make the following
observations about $\alpha\left(s\right)$. First, $\underset{\kappa\rightarrow0}{lim}\alpha\left(s\right)=\frac{\beta^{\frac{-1}{\rho}}}{1+\beta^{\frac{-1}{\rho}}}$
(this follows from the fact that at $\kappa=0$, the contract must
satisfy $u'\left(c_{1}\right)=\beta u'\left(c_{2}\right)$). Second,
implicitly differentiating equation \ref{eq:alpha} with respect to
$s$, and combining it with the previous limit result, we get $\underset{\kappa\rightarrow0}{lim}\frac{d\alpha\left(s\right)}{ds}=0$.
Therefore, if $\rho>1$ and $\kappa$ is small enough, the second
term in Equation \ref{eq:dv-ds-comp} will be sufficiently small in
magnitude that $\frac{dV\left(s^{F},\alpha\left(s^{F}\right)\right)}{ds}>\frac{dV\left(s^{F},\frac{1}{2}\right)}{ds}=\frac{du\left(c_{0}^{F}\right)}{dc}$.
In this case, the renegotiation-proof contract must satisfy $c_{0}^{P}<c_{0}^{F}$.
$\Square$

If $\kappa=0$, the renegotiation-proof contracts can be explicitly
derived since in any contract it must be true that $c_{2}=\beta^{\frac{1}{\rho}}c_{1}$.
Solving the respective maximization problems, we get the following
equilibrium contracts for monopoly and competition, respectively:
\begin{align}
C_{0}^{mP}= & \left(\left(\frac{U_{0}^{A}\left(1-\rho\right)}{1+\beta^{\frac{1}{\rho}}\left(\frac{\left(1+\beta^{\frac{1-\rho}{\rho}}\right)^{\frac{1}{\rho}}}{\left(1+\beta^{\frac{1}{\rho}}\right)^{\frac{1-\rho}{\rho}}}\right)}\right)^{\frac{1}{1-\rho}},\left(\frac{\beta+\beta^{\frac{1}{\rho}}}{1+\beta^{\frac{1}{\rho}}}\right)^{\frac{1}{\rho}}c_{0}^{mP},\beta^{\frac{1}{\rho}}\left(\frac{\beta+\beta^{\frac{1}{\rho}}}{1+\beta^{\frac{1}{\rho}}}\right)^{\frac{1}{\rho}}c_{0}^{mP}\right)\label{eq:zerokappa-monop}\\
C_{0}^{P}= & \left(\frac{y}{1+\beta+\beta^{\frac{1}{\rho}}},\left(\frac{\beta+\beta^{\frac{1}{\rho}}}{1+\beta^{\frac{1}{\rho}}}\right)c_{0}^{P},\beta^{\frac{1}{\rho}}\left(\frac{\beta+\beta^{\frac{1}{\rho}}}{1+\beta^{\frac{1}{\rho}}}\right)c_{0}^{P}\right)\label{eq:zerokappa-comp}
\end{align}

It can easily be established that $c_{0}^{mP}>c_{0}^{mF}$, $c_{0}^{P}>c_{0}^{mF}$
if $\rho>1$, and $c_{0}^{P}<c_{0}^{mF}$ if $\rho<1$.

\subsubsection{Naive Hyperbolic Discounters}

Suppose the monopolist intends to renegotiate the contract. The maximization
problem, combined with the expression for $C_{1}^{m1}\left(C_{1}\right)$
(\ref{eq:c-r-alpha}), simplifies to:

\begin{align}
\underset{c_{0},c_{1},c_{2}}{max} & y-c_{0}-\frac{\left(c_{1}^{1-\rho}+\beta c_{2}^{1-\rho}\right)^{\frac{1}{1-\rho}}}{\left(1+\beta^{\frac{1}{\rho}}\right)^{\frac{\rho}{1-\rho}}}-\kappa\\
s.t. & \frac{c_{0}^{1-\rho}}{1-\rho}+\beta\frac{c_{1}^{1-\rho}}{1-\rho}+\beta\frac{c_{2}^{1-\rho}}{1-\rho}\geq U_{0}^{A}
\end{align}

The partial derivatives of the resulting Lagrangian are: 
\begin{align}
\frac{\partial\mathcal{L}}{\partial c_{0}} & =-1-\lambda c_{0}^{-\rho}\label{eq:L0}\\
\frac{\partial\mathcal{L}}{\partial c_{1}} & =c_{1}^{-\rho}\left[-\left(\frac{c_{1}^{1-\rho}+\beta c_{2}^{1-\rho}}{1+\beta^{\frac{1}{\rho}}}\right)^{\frac{\rho}{1-\rho}}-\lambda\beta\right]\label{eq:L1}\\
\frac{\partial\mathcal{L}}{\partial c_{2}} & =c_{2}^{-\rho}\left[-\beta\left(\frac{c_{1}^{1-\rho}+\beta c_{2}^{1-\rho}}{1+\beta^{\frac{1}{\rho}}}\right)^{\frac{\rho}{1-\rho}}-\lambda\beta\right]\label{eq:L2}
\end{align}

An interior solution, with $\frac{\partial\mathcal{L}}{\partial c_{1}}=0$
and $\frac{\partial\mathcal{L}}{\partial c_{2}}=0$ does not exist
(on a $c_{1}-c_{2}$ plot, the two first-order conditions do not intersect).
If $\rho<1$, the Lagrangian is maximized at a corner solution with
$c_{1}=0$. If $\rho>1$, the Lagrangian is maximized at the limit
as $c_{2}$ approaches infinity. Using this, the maximization problem
can be re-solved. If $\rho<1$: 
\begin{equation}
C_{0}^{mN}=\left(\left(\frac{U_{0}^{A}\left(1-\rho\right)}{2+\beta^{\frac{1}{\rho}}}\right)^{\frac{1}{1-\rho}},0,\left(\frac{1+\beta^{\frac{1}{\rho}}}{\beta}\right)^{\frac{1}{1-\rho}}\left(\frac{U_{0}^{A}\left(1-\rho\right)}{2+\beta^{\frac{1}{\rho}}}\right)^{\frac{1}{1-\rho}}\right)\label{eq:naive-monopolist-contract1}
\end{equation}
If $\rho>1$, the solution is undefined, but in the limit is given
by: 
\begin{equation}
C_{0}^{mN}=\left(\left(\frac{U_{0}^{A}\left(1-\rho\right)}{1+\left(1+\beta^{\frac{1}{\rho}}\right)\beta^{\frac{1}{\rho}}}\right)^{\frac{1}{1-\rho}},\beta^{\frac{1}{\rho}}\left(1+\beta^{\frac{1}{\rho}}\right)^{\frac{1}{1-\rho}}\left(\frac{U_{0}^{A}\left(1-\rho\right)}{1+\left(1+\beta^{\frac{1}{\rho}}\right)\beta^{\frac{1}{\rho}}}\right)^{\frac{1}{1-\rho}},\infty\right)\label{eq:naive-monopolist-contract2}
\end{equation}

\textbf{\emph{Proof of Proposition 4:}} (a) At any autarky utility,
the monopolist could at least offer the full-commitment contract.

(b) The bank must choose between a renegotiation-proof contract and
a renegotiable contract (\ref{eq:naive-monopolist-contract1}, \ref{eq:naive-monopolist-contract2}).
By construction of $U^{m}$, the following must be true at any $U_{0}^{A}\geq U^{m}$:
\begin{equation}
\Pi_{0}\left(C_{0}^{mP};Y_{0}\right)\leq\Pi_{0}\left(C_{0}^{mF};Y_{0}\right)\leq\Pi_{0}\left(C_{0}^{mF};Y_{0}\right)+\Pi_{1}\left(C_{1}^{m1}\left(C_{1}^{mF}\right);C_{1}^{mF}\right)-\kappa
\end{equation}

Since $C_{0}^{mN}$ is uniquely determined and $C_{0}^{mN}\neq C_{0}^{mF}$,
profits from the best renegotiable contract must be strictly higher
than profits from the renegotiation-proof contract at any $U_{0}^{A}\geq U^{m}$.

The following can be verified from the explicit derivations of $C_{0}^{mF}$
and $C_{0}^{mN}$. First, if $U_{0}^{A}$ is sufficiently small, $\Pi_{0}\left(C_{0}^{mF};Y_{0}\right)>\Pi\left(C_{0}^{mN};Y_{0}\right)+\Pi_{1}\left(C_{1}^{m1}\left(C_{1}^{mN}\right);C_{1}^{mN}\right)-\kappa$.
Second, 
\begin{equation}
\frac{d}{dU_{0}^{A}}\Pi_{0}\left(C_{0}^{mF};Y_{0}\right)>\frac{d}{dU_{0}^{A}}\left[\Pi\left(C_{0}^{mN};Y_{0}\right)+\Pi_{1}\left(C_{1}^{m1}\left(C_{1}^{mN}\right);C_{1}^{mN}\right)-\kappa\right]
\end{equation}

It follows that there is some $U^{N}<U^{m}$ such that if $U_{0}^{A}\leq U^{N}$,
the naive agent will receive the monopoly full commitment contract,
which will be renegotiation-proof.

(c) This can be confirmed from the explicit formulations of $C_{0}^{mF}$
(\ref{eq:c-mf}) and $C_{0}^{mN}$ (\ref{eq:naive-monopolist-contract1},
\ref{eq:naive-monopolist-contract2}). $\Square$

We now derive equilibrium contracts for naive consumers under perfect
competition. Suppose contracts are exclusive. Then, a contract that
is renegotiated satisfies: 
\begin{align}
\underset{c_{0},c_{1},c_{2}}{max} & \frac{c_{0}^{1-\rho}}{1-\rho}+\beta\frac{c_{1}^{1-\rho}}{1-\rho}+\beta\frac{c_{2}^{1-\rho}}{1-\rho}\\
s.t. & y-c_{0}-\frac{\left(c_{1}^{1-\rho}+\beta c_{2}^{1-\rho}\right)^{\frac{1}{1-\rho}}}{\left(1+\beta^{\frac{1}{\rho}}\right)^{\frac{\rho}{1-\rho}}}-\kappa\geq0
\end{align}

The first-order conditions are the same as under monopoly (\ref{eq:L0},
\ref{eq:L1}, \ref{eq:L2}). Combining these with the zero-profit
constraint, we get the following solution. If $\rho<1$: 
\begin{equation}
C_{0}^{N}=\left(\frac{y-\kappa}{2+\beta^{\frac{1}{\rho}}},0,\left(\frac{1+\beta^{\frac{1}{\rho}}}{\beta}\right)^{\frac{1}{1-\rho}}\left(\frac{y-\kappa}{2+\beta^{\frac{1}{\rho}}}\right)\right)\label{eq:naive-comp-contract1}
\end{equation}

If $\rho>1$, the solution is undefined, but in the limit is given
by: 
\begin{equation}
C_{0}^{N}=\left(\frac{y-\kappa}{1+\beta^{\frac{1}{\rho}}\left(1+\beta^{\frac{1}{\rho}}\right)},\beta^{\frac{1}{\rho}}\left(1+\beta^{\frac{1}{\rho}}\right)^{\frac{1}{1-\rho}}\left(\frac{y-\kappa}{1+\beta^{\frac{1}{\rho}}\left(1+\beta^{\frac{1}{\rho}}\right)}\right),\infty\right)\label{eq:naive-comp-contract2}
\end{equation}

\textbf{\emph{Proof of Proposition 5:}} (a) Banks can at least offer
the consumer the full-commitment contract, so a contract is feasible
at any autarky utility. Since, from the consumer's perspective, any
renegotiation-proof contract is strictly dominated by the full-commitment
contract (which will be renegotiated), in equilibrium she will be
offered a contract that will be renegotiated.

(b) Under non-exclusive contracts, firms offering period 0 contracts
do not benefit from renegotiation (profits from renegotiation will
equal $\kappa$). So the equilibrium contract is identical to the
full-commitment contract.

(c) Suppose $\rho<1$. Comparing $C_{0}^{F}$ (\ref{eq:c-cf}) to
$C_{0}^{N}$ (\ref{eq:naive-comp-contract1}), it is clear that $c_{0}^{N}<c_{0}^{F}$.
Suppose $\rho>1$. If $\kappa$ is small enough, $c_{0}^{N}>c_{0}^{F}$.
$\Square$

\subsection{Nonprofits}
\begin{lem}
$\Pi_{0}\left(C_{0}^{mP};Y_{0}\right)$ and $\bar{\alpha}\Pi_{0}\left(C_{0}^{m\alpha};Y_{0}\right)$
are continuously decreasing in $U_{0}^{A}$. 
\end{lem}
\textbf{\emph{Proof of Lemma 1: }}We prove the above for renegotiation-proof
contracts of for-profit banks. The same argument applies to nonprofit
banks. First, it is clear that profits are strictly decreasing in
$U_{0}^{A}$: If autarky utility drops from $U_{0}^{A}=U$ to $\bar{U_{0}^{A}=U}$,
at $\bar{U}$ the bank can always do better than offering the contract
it offered at $U$.

Next, we prove right-continuity at any $U_{0}^{A}=U$. Let the maximized
profits at $U$ be $\Pi_{0}\left(C_{0};Y_{0}\right)$, where $C_{0}=\left(c_{0},c_{1},c_{2}\right)$.
This contract must satisfy $U_{0}\left(C_{0}\right)=U$. For any $\bar{U}>U$,
profits must be lower, and bounded below by $\Pi_{0}\left(\bar{C}_{0};Y_{0}\right)$,
with the contract defined as $\bar{C}_{0}=\left(c_{0}+x,c_{1},c_{2}\right)$
where $x$ satisfies $U_{0}\left(\bar{C}_{0}\right)=\bar{U}$. Since
$\underset{\bar{U}\rightarrow U^{+}}{lim}\Pi_{0}\left(\bar{C}_{0};Y_{0}\right)=\Pi_{0}\left(C_{0};Y_{0}\right)$,
the profit function is right-continuous.

Finally, we prove left-continuity at at any $U_{0}^{A}=U$. For any
$\bar{U}<U$, denote maximized profits $\Pi_{0}\left(\bar{C}_{0};Y_{0}\right)$,
where $\bar{C}_{0}=\left(\bar{c}_{0},\bar{c}_{1},\bar{c}_{2}\right)$.
These contracts must satisfy $U_{0}\left(\bar{C}_{0}\right)=\bar{U}$.
At $U$, profits must be lower, and bounded below by $\Pi_{0}\left(C_{0};Y_{0}\right)$,
with the contract defined as $C_{0}=\left(\bar{c}_{0}+x,\bar{c}_{1},\bar{c}_{2}\right)$
where $x$ satisfies $U_{0}\left(C_{0}\right)=U$. Since $\underset{\bar{U}\rightarrow U^{-}}{lim}\Pi_{0}\left(\bar{C}_{0};Y_{0}\right)=\Pi_{0}\left(C_{0};Y_{0}\right)$,
the profit function is left-continuous. $\Square$

\textbf{\emph{Proof of Proposition 6: }}Let $\eta\left(\bar{\alpha}\right)=\kappa$,
to minimize the attractiveness of the nonprofit. Consider $U_{0}^{A}=\bar{U}$.
Since the for-profit's renegotiation-proofness binds and leaves the
firm with zero profits, and since the non-profit's renegotiation-proofness
constraint is looser than the for-profit's, we know that $\bar{\alpha}\Pi\left(C_{0}^{m\alpha};Y_{0}\right)>0=\Pi\left(C_{0}^{mP};Y_{0}\right)$.
Since profits must be continuously decreasing in $U_{0}^{A}$, and
since $\bar{\alpha}\Pi\left(C_{0}^{m\alpha};Y_{0}\right)<\Pi\left(C_{0}^{mP},;Y\right)$
at $U_{0}^{A}=U^{m}$ (where the for-profit's renegotiation-proofness
constraint no longer binds) and $\bar{\alpha}\Pi\left(C_{0}^{m\alpha};Y_{0}\right)\leq0$
at $U_{0}^{A}=U_{0}^{F}$, there must exist autarky utility values
as described in the proposition statement such that, if $U_{0}^{A}>\underline{U}^{\alpha}$
the bank strictly prefers to operate as a nonprofit relative to a
for-profit, and if $U_{0}^{A}\geq\bar{U}^{\alpha}$ it weakly prefers
to not offer a contract. $\Square$

\textbf{\emph{Proof of Proposition 7: }}(a) Suppose all firms are
for-profit. There is some $\varepsilon_{1}$ and $\varepsilon_{2}$
satisfying $0<\varepsilon_{2}<\varepsilon_{1}$ and a corresponding
$\hat{C}_{0}=\left(c_{0}^{\alpha},c_{1}^{\alpha}-\varepsilon_{1},c_{2}^{\alpha}+\varepsilon_{2}\right)$
such that $U_{0}\left(C_{0}^{\alpha}\right)=U_{0}\left(\hat{C}_{0}\right)$
and $\Pi_{0}(C_{1}^{m1}\left(\hat{C}_{1}\right);\hat{C}_{1})\leq\kappa(\mbox{\ensuremath{\bar{\alpha}}})<\kappa\left(1\right)$.
So, any firm can make positive profits by operating as a non-profit.
Therefore, in equilibrium, consumers will borrow only from non-profit
firms. Given the construction of $\bar{U}^{\alpha}$, firms can make
nonnegative profits while satisfying the participation constraint
only if $U_{0}^{A}\leq\bar{U}^{\alpha}$.

(b) If all firms are nonprofit, an individual firm has a strict incentive
to switch to for-profit status, and make profits in period 1. Therefore,
there must be for-profits in equilibrium, and equilibrium contracts
will be constrained by their presence. $\Square$

\bibliographystyle{authordate1}
\bibliography{renegotiation}

\end{document}
