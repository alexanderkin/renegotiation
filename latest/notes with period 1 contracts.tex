%% LyX 2.2.0 created this file.  For more info, see http://www.lyx.org/.
%% Do not edit unless you really know what you are doing.
\documentclass[english]{article}
\usepackage[T1]{fontenc}
\usepackage[latin9]{inputenc}
\usepackage{amsmath}

\makeatletter
%%%%%%%%%%%%%%%%%%%%%%%%%%%%%% User specified LaTeX commands.
% set fonts for nicer pdf view
\IfFileExists{lmodern.sty}{\usepackage{lmodern}}{}

\makeatother

\usepackage{babel}
\begin{document}

Our preceding analysis made a simplifying assumption that contracts
between consumers and banks could only be initiated in period 0. This
served to streamline the analysis, as the alternative to a period
0 contract was autarky (for the consumer) and zero profits (for the
bank). We now discuss how the problem changes if unbanked consumers
are permitted to sign two-period contracts in period 1.

The main change relates to the formulation of reservation values and
their implications for the shape and feasibility of the period 0 contract.
Consider the monopolist bank facing a sophisticated hyperbolic discounter.
If a contract were not signed in period 0, they would meet again in
period 1. In period 1, the contract must satisfy the One-self's participation
constraint, which would be determined by some unbanked consumption
path $C_{1}^{A'}$. This can be stated formally. Given some autarky
consumption path $C_{1}^{A'}$, the bank solves:
\begin{align*}
\underset{C_{1}}{max} & \Pi_{1}\left(C_{1};C_{1}^{A'}\right)\\
\text{s.t.} & \text{\ensuremath{U_{1}\left(C_{1}\right)\geq U_{1}\left(C_{1}^{A'}\right)}}
\end{align*}

Let the solution be denoted $C_{1}^{m1}$. Two observations can be
made. First, the bank can always offer a period 1 contract that delivers
nonnegative profits. This is because any contract will satisfy One-self's
optimality condition, $u'\left(c_{1}^{m1}\right)=\beta u'\left(c_{2}^{m1}\right)$.
So, except in the special case where the autarky consumption path
satisfies this condition, the bank can make positive profits in period
1. Second, the autarky consumption path $C_{1}^{A'}$ might differ
from $C_{1}^{A}$, the consumer's autarky utility in the absence of
banking. In other words, $C_{1}^{A}$ maximizes $U_{0}\left(C_{0}^{A}\right)$
while $C_{1}^{A'}$ maximizes $U_{0}\left(c_{0}^{A'},c_{1}^{m1},c_{2}^{m2}\right)$.
In the latter case, period 0 anticipates that consumption across periods
1 and 2 is guaranteed to satisfy period 1's optimality condition.
We can denote $C_{0}^{B}\equiv\left(c_{0}^{A'},c_{1}^{m1},c_{2}^{m2}\right)$,
which corresponds to a Zero-self utility of $U_{0}^{B}$.

In period 0, any contract must meet the Zero-self's reservation utility,
$U_{0}^{B}$:
\begin{align*}
\underset{C_{0}}{max} & \Pi_{0}\left(C_{0};Y_{0}\right)\\
\text{s.t.} & \text{\ensuremath{U_{0}\left(C_{0}\right)\geq U_{0}^{B}}}
\end{align*}

The maximization problem looks familiar, apart from the modified reservation
utility. The Zero-self's discounted utility from such a contract is
no longer monotonic in her full autarky utility, $U_{0}^{A}$. For
example, consider two hypothetical consumers who in autarky must consume
their income streams, which deliver the same autarky utility but through
different consumption paths: consumer X has $c_{1}^{A}=c_{2}^{A}$
while consumer Y has $c_{1}^{A}>c_{2}^{A}$ in a way that satisfies
period 1's optimality condition. Then, for consumer X, $U_{0}^{B}<U_{0}^{A}$
while for consumer Y, $U_{0}^{B}=U_{0}^{A}$. It follows that, since
period 0 contracts depend on the distribution of future consumption,
a consumer who fares relatively better in the absence of a bank may
fare relatively worse under a banking contract.

Given this benchmark full-commitment contract, the renegotiation-proof
contract can be solved for by adding a no-renegotiation constraint
to the above maximization problem. The constraint is the same as used
previously, and again narrows the set of contracts that can be offered
in period 0. As in Proposition 2 (parts a and c), the renegotiation-proof
constraint results in lower profits and greater period 0 consumption
relative to full-commitment. These results are independent of the
period 0 reservation utility and therefore remain unchanged.

A key difference here, however, is that a renegotiation-proof contract
will be offered to all consumers (unlike before, where the bank was
better off not contracting with consumers whose autarky utility left
them close enough to the first-best). Intuitively, this is because
the alternative to a period 0 contract is not autarky; rather, it
is a period 1 contract that tilts consumption in period 1's favor.
Since, in period 0, the bank can at least offer the consumer a consumption
path of $C_{0}^{B}$, it ensures that a contract will be accepted.

By opening up the possibility of period 1 contracts, we introduce
an additional consideration\textendash the bank that offers commitment
itself creates a need for it. By threatening to fully indulge the
One-self's preferences, the bank is always able to induce the Zero-self
to accept an offer of partial commitment, no matter how weak.

Finally, observe that the bank's decision about whether to operate
as a nonprofit is subject to the same tradeoff between improved commitment
and reduced enjoyment of profits. However, the attractiveness of nonprofit
status drops (relative to the case where period 1 contracts are disallowed)
due to the fact that even the for-profit bank finds it profitable
to offer contracts to consumers at all levels of autarky utility.

Next, we turn to competition. In most cases, the possibility of period
1 contracts leaves our previous analysis unaltered. This is because
competitive contracts do not depend on autarky utility. The only modification
to our previous results relates to Proposition 3 (part b). Now, the
consumer will accept a period 0 renegotiation-proof contract at any
autarky utility since not doing so exposes her to a period 1 contract
that fully satisfies period 1's taste for imbalanced consumption.
\end{document}
