%% LyX 2.1.3 created this file.  For more info, see http://www.lyx.org/.
%% Do not edit unless you really know what you are doing.
\documentclass[english]{article}
\usepackage[T1]{fontenc}
\usepackage[latin9]{inputenc}
\usepackage{amsthm}
\usepackage{amsmath}
\usepackage{graphicx}

\makeatletter
%%%%%%%%%%%%%%%%%%%%%%%%%%%%%% Textclass specific LaTeX commands.
  \theoremstyle{plain}
  \newtheorem{lem}{\protect\lemmaname}
  \theoremstyle{plain}
  \newtheorem{prop}{\protect\propositionname}

\makeatother

\usepackage{babel}
  \providecommand{\lemmaname}{Lemma}
  \providecommand{\propositionname}{Proposition}

\begin{document}

\section{Introduction}

{[}Placeholder{]}


\section{The Model: Assumptions}


\subsection{Consumers}

There are three periods, $t\in\left\{ 0,1,2\right\} $. In any period
$t$, the consumer's instantaneous utility is given by $u\left(c_{t}\right)$
for $c_{t}\in$ $[0,\infty)$. The utility function is twice differentiable,
strictly concave, and satisfies $u^{\prime}\left(0\right)=\infty$.
Given a consumption stream $C_{t}=\left(c_{t},c_{t+1},...,c_{2}\right)$,
the period-$t$ self's discounted utility is: 
\[
U_{t}\left(C_{t}\right)\equiv u\left(c_{t}\right)+\beta\sum\limits _{i=t+1}^{2}\delta^{i-t}u\left(c_{i}\right)
\]
This describes quasi-hyperbolic preferences, with a hyperbolic discount
factor $\beta\in(0,1]$.\footnote{More generally, utility would also be discounted by an exponential
discount factor $\delta$, so that $U_{t}\left(C_{t}\right)\equiv u\left(c_{t}\right)+\beta\sum\limits _{i=t+1}^{2}\delta^{i-t}u\left(c_{i}\right)$.
To streamline the analysis without loss of generality, we assume $\delta=1$} In any period, the individual places greater relative weight on present-period
gratification than his earlier selves did. The consumer could be sophisticated,
naive, or partially naive about the time-inconsistency of his preferences
(O'Donoghue \& Rabin, 2001). Our analysis will focus on the sophisticated
agent, with extended discussions to other types.

The period-0 consumer starts with claims to an existing arbitrary
income stream $Y_{0}=\left(y_{0},y_{1},y_{2}\right)$ over the three
periods, but will in general be interested in consumption via 'own-savings'
or financial contracting opportunities with others. If banking or
own-savings technologies are absent or prohibitively expensive, the
consumer's autarky consumption stream $C_{0}^{a}(Y)=\left(c_{0}^{a},c_{1}^{a},c_{2}^{a},\right)$
is just the exogenous income stream $Y.$ 

Depending on the initial income stream, with available own-savings
strategies the consumer may be able to construct a somewhat better
smoothed autarky stream. However, as a consumer with time-inconsistent
preferences cannot trust his later selves to not undo his preferred
consumption choices, the stream will be the imperfectly-smoothed outcome
of a strategic game between selves as explained in Basu (2014). This
creates the demand for commitment services via contracts with an outside
financial intermediary, and this is the main focus of our analysis.
\ For the present moment we take this autarky option $C_{0}^{a}(Y)$
to be fixed and given, but later show how the main results are only
slightly modified when we treat it as endogenous.\footnote{Since a sophisticated hyperbolic discounter is fully aware of his
future preferences, his autarky consumption stream will be a subgame
perfect equilibrium of the consumption-savings game played by the
series of his time-indexed selves (each with a slightly different
objective function).} 

In the sections that follow we state our results for quite general
assumptions about the shape of the the instantaneous utility function
$u(c)$ but illustrate and provide exact results for the CRRA case
$u(c_{t})=\frac{c^{1-\rho}}{1-\rho}$ where parameter $\rho$ is closely
related to the elasticity of intertemporal substitution. The appendix
contains closed form solutions for the CRRA case and other technical
details.


\subsection{Banks}

The period-0 consumer has the option of contracting with one or many
banks, depending on the market structure. Each bank has access to
funds that can be withdrawn from other investments at opportunity
cost $r$. For expositional simplicity in main text results below
we assume $r=0$ and that the consumer's discount factor is $\delta=\frac{1}{1+r}$
(which implies $\delta=1)$ but this is without loss of generality
and more formulas for arbitrary $r$ and $\delta$ are in the appendix.
Any financial contract between the consumer and the bank can be thought
of as the consumer giving up income stream $Y_{0}$ in exchange for
a weakly preferred consumption path $C_{0}=(c_{0},c_{1},c_{2})$.
A financial contract involves a period 0 loan if the exchange is such
that $c_{0}>y_{0}$ and a savings contract if $c_{0}<y_{0}$. For
analytical convenience, we initially assume that any new contract
must be initiated in period 0. 

The bank can perfectly enforce feasible repayment plans, so the net
present value of its remaining profit stream in period t is given
by:
\[
\pi_{t}\left(C_{t};Y_{t}\right)=\sum\limits _{i=t}^{2}\left(y_{t}-c_{t}\right)
\]


The bank offers a contract if and only if it can earn non-negative
profits.

If the contract were to be renegotiated in period 1, the bank will
incur a non-monetary cost, $\bar{\kappa}$, which could be interpreted
to include a concern for reputation or some other impact on the social
preferences of its owners.\footnote{The bank could incur additional monetary costs as well. However, we
assume these to be 0 as they can be netted out and do not affect the
analysis in any important way.} The bank fully knows the borrowers preferences and can tailor the
contract by borrower type.

While the primary focus of this paper is on equilibrium contracts,
not welfare, we will make occasional reference to the hyperbolic discounter's
welfare defined as the discounted utility of the period 0 self, $U_{0}$
(for a discussion the choice of welfare measure see O'Donoghue-Rabin,
1999).


\section{Monopolist Bank}

In this section, we analyze the renegotiation problem when a sophisticated
present-biased consumer faces a monopolist lender. We first derive
properties of the benchmark equilibrium 'full-commitment contract'
in the absence of any renegotiation possibilities. We then describe
how the contract would change when the consumer understands and anticipates
the possibility of contract renegotiation between his later-period
self and the bank. Finally, we describe conditions under which a lender
might strategically choose to operate as a commercial non-profit.


\subsection{Full-Commitment Contract}

The monopolist who can offer a full-commitment contract solves the
following problem:

\[
\max_{C_{0}}\pi_{0}\left(C_{0};Y_{0}\right)
\]


\[
\text{s.t. (PC) }U_{0}\left(C_{0}\right)\geq U_{0}\left(C_{0}^{a}(Y_{0})\right)
\]


Let the monopolist's full-commitment contract be denoted $C^{mf}$.
At an optimum: 
\begin{align}
u^{\prime}\left(c_{0}^{mf}\right)=\beta u^{\prime}\left(c_{1}^{mf}\right)=\beta u^{\prime}\left(c_{2}^{mf}\right)\label{foc-monop}\\
U_{0}\left(C_{0}^{mf}\right)=U_{0}\left(C_{0}^{a}(Y_{0})\right)\label{BPC-monop}
\end{align}


Condition \ref{foc-monop} is the standard first-order necessary condition
that marginal utilities of consumption be equalized across periods
which, for our simplifying assumptions, imply that $c_{1}^{mf}=c_{2}^{mf}=\overline{c}^{mf}$.
Condition \ref{BPC-monop} is the consumer participation constraint
that they be no worse off by accepting the contract than under autarky.

For the CRRA\ case the FOC imply $\overline{c}^{mf}=\beta^{\frac{1}{\rho}}c_{0}^{mf}$
which, along with the participation constraint, allow for closed form
solutions (see appendix). See Figure 1.

The terms of the optimal contract, and hence also the level of bank
profits, will depend on the consumer's starting income stream $Y_{0}$.
To see this, consider the set of possible initial income streams of
constant present value $\sum_{t=0}^{2}y_{t}=\bar{y}$. We denote the
set of possible autarky income streams $\varOmega$. This allows us
to consider a range of cases where the consumer has a constant lifetime
wealth that is possibly sub-optimally distributed. The banking problem
therefore becomes a consumption-smoothing problem for indivuals of
varying distributions of wealth.

In Figure 1, the 3-dimensional grey triangle describes the set of
possible autarky consumption bundles. The bank's objective is to replace
the autarky consumption bundle with one that is on the lowest possible
isoprofit plane (which denotes the highest possible profits) while
being acceptable to the consumer. To make positive profits, it must
be able to offer a consumption bundle that lies closer to the origin
than the autarky bundle. Of the autarky bundles, $C_{0}^{a*}$is already
optimal in the sense that it already satisfies $u^{\prime}\left(c_{0}^{a}\right)=\beta u^{\prime}\left(c_{1}^{a}\right)=\beta u^{\prime}\left(c_{2}^{a}\right)$.
However, any other autarky consumption bundle is associated with a
curved 'indifference surface' that dips below the grey triangle. So,
the bank can offer a consumption bundle that brings discounted marginal
utilities closer together, earning positive profits in the process.
Whether the new bundle constitutes a loan or savings contract depends
on the position of the consumer's autarky bundle.

\includegraphics[scale=0.5]{fig1}


\subsection{The Renegotiation Problem}

Bank profits can be thought of as deriving in part from satisfying
the customer's demand for commitment. That is to say, the bank can
charge for helping customers to stick to preferred saving accumulation
\textit{cum} loan repayment plans, that in turn deliver balanced consumption
across periods 1 and 2. However, this is conditional on its ability
to credibly promise to not pander to the customers' later self's demands
to raid savings and/or take out new loans that would disrupt those
plans.

Figure 2 which is drawn in $c_{1}-c_{2}$ space illustrates why the
demand for commitment arises, how a monopolist could profit by reneging
on such a commitment \textit{ex-post} and therefore why a monopolist
might take costly actions to try to make such commitments more credible
\textit{ex-ante}. Point $F$ is the balanced consumption $\left(\bar{c}^{mf},\mbox{\ensuremath{\overline{c}}}^{mf}\right)$
component of the monopolist's optimal contract. At this point period
0 consumption is $c_{0}^{mf}$ and the participation constraint (\ref{BPC-monop})
is satisfied. The flatter indifference curve passing through point
$F$ represents the consumer's period 0 self's preference, $u\left(c_{1}\right)+u\left(c_{2}\right)$,
which gives equal weight to period 1 and period 2 consumption (in
period 1 utils) while line NF can be interpreted as the isoprofit
line associated with the monsopolist's maximum profits from period
1 forward.

The demand for commitment arises from the fact that the period 0 self
understands that as soon as period 1 rolls around his preferences
will change and the period 1 self will now accept renegotiated contract
offers to raise period 1 consumption even if this comes at the cost
of period 2 consumption and period 0's discounted utility.\footnote{Measured in period 1 utils, the period 0 self values consumption bundle
$(c_{1},c_{2})$ as $u(c_{1})+u(c_{2})$ whereas period 1 self values
the same bundle as $u(c_{1})+\beta u(c_{2})$.}. Hence at contract point $F$ in the figure we have 
\[
u^{\prime}\left(c_{1}^{mf}\right)\geq\beta u^{\prime}\left(c_{2}^{mf}\right)
\]


If he is a hyperbolic discounter with $\beta<1$, he wishes to increase
present consumption. As the figure illustrates, a monopolist can opportunistically
'exploit' a consumer who had agreed to $F$ by offering consumption
stream $R$ in exchange for $F$. This would raise period 1 consumption
by 'raiding savings' (taking out new loans), for a customer who had
saved (borrowed) in period 0 by with the effect of lowering (increasing)
the net savings (debt) passed into period 2.

FIGURE: Monopoly full-commitment contract in $c_{1}$$-c_{2}$ space.

Such a contract would be designed to raise period 1 self's utility
ever so slightly while raising the monopolist to the higher value
isoprofit line $QN$.

This creates an opportunity for the bank to generate additional surplus.
The bank might be willing to offer such a refinance, in effect offering
a new loan and/or postponing repayments on existing obligations or
-- if the customer had carried savings into period 1 -- helping the
customer 'raid' savings for a fee. In such cases, both parties voluntarily
agree to renegotiate the original contract.

A sophisticated present-biased customer anticipates such potential
renegotiations and will therefore only agree to renegotiation-proof
contracts. Since making contracts renegotiation proof only adds constraints
to the contract design problem the monopolists' profits can only fall
relative to the benchmark where it can costlessly commit itself to
no renegotiations. The customer who was already pushed against their
autarky reservation utility is no worse off.

More formally, consider the renegotiation problem that arises in period
1. A customer who has agreed to contract $\hat{C}\left(0\right)$
in period 0 enters period 1 with claims to remaining consumption stream
$\hat{C}\left(1\right)$. This determines the new reservation utility.
If profitable to the bank, the optimal renegotiated contract it offers
will be given by:

\[
\max\pi_{1}\left(C_{1};\hat{C}_{1}\right)
\]


\[
\text{s.t. }\text{}U_{1}\left(C_{1}\right)\geq U_{1}\left(\hat{C}_{1}\right)
\]


The bank offers $C_{1}^{r}\left(C_{1}\right)=\left(c_{1}^{r},c_{1}^{r}\right)$
to replace\footnote{The bank swaps net income stream $\left(y_{1}-\hat{c}_{1},y_{2}-\hat{c}_{2}\right)$
for $(y_{1}-c_{1}^{r},y_{2}-c_{2}^{r})$.} the contractually agreed $(\hat{c}_{1},\hat{c}_{2}).$ Since the
first-order conditions imply: 
\[
u^{\prime}\left(c_{1}^{r}\right)=\beta u^{\prime}\left(c_{2}^{r}\right)
\]
the contract will set $c_{1}^{r}>$ $c_{2}^{r}$ as long as $\beta<1$
and exactly $c_{2}^{r}=\beta^{1/\rho}c_{1}^{r}$ in the CRRA case.
FIGURE illustrates the type of renegotiated contract that would be
offered in the very special case of a customer who agreed to a full
commitment contract $C_{1}^{mf}$ in period 0 -- having sincerely
believed the bank's promise to not offer to renegotiate in period
1 -- and was then genuinely and unexpectedly surprised to find the
bank offering a renegotiation in period 1. Although it would break
a promise to the consumer's period 0 self, the contract will be designed
to appeal to the customer's period 1. This type of renegotiation/promise-breaking
will be profitable for the bank to offer as long as $\pi_{1}\left(C_{1}^{r}\left(C_{1}^{mf}\right);C_{1}^{mf}\right)\geq\kappa$,
where $\kappa$ are the non-monetary renegotiation costs to the bank.
As depicted in Figure FIGURE, the bank will find it profitable to
renegotiate from the continuation of the full commitment contract
depicted by point $F$ to the new contract depicted by $B$ so long
as profits gains (equal to the present value difference between $C_{1}^{mf}$
and $C_{1}^{r}$) which can be measured as distance $NQ$ on the diagram
is greater than $\kappa$. Although the new contract appeals to the
period 1 self, the broken promise 'exploits' the consumer's period
0 self by pushing their welfare below the autarky reservation utility
that they thought had been promised by the $C_{1}^{mf}$ contract.


\subsection{Renegotiation-Proof Contracts}

A sophisticated consumer will, of course, rationally anticipate the
bank's temptation to break its commitment promises and for this reason
would never agree to accept contract $F$ or any other contract that
could make him worse off following a period-1 renegotiation. Since
the bank will want to avoid renegotiation costs $\kappa$ the only
contracts that will be offered that satisfy this criterion in period
0 must be renegotiation-proof contracts: contracts that the bank will
not find profitable to renegotiate.

The bank must therefore offer a contract that satisfies period 0's
participation constraint and now also meets an additional renegotiation-proofness
constraint. The resulting contract must limit period 1 self's (and
hence also the bank's) potential gains from renegotiation. It does
this by increasing $c_{1}$by just enough relative to $c_{2}$as to
lower what the period 1 self is willing to pay for a renegotiation
to fall just below the bank's cost $\kappa$, as to keep renegotiation
credibly unprofitable to the bank. If $\kappa$ is small enough, this
must lower bank profits relative to the costless full-commitment case
because the monopolist must now compensate the consumer for a contract
that delivers less consumption smoothing across period 1 and 2 than
under full commitment.

The now solves a maximization problem with two constraints.

\begin{align*}
 & \max_{C_{0}}\pi_{0}\left(C_{0};Y_{0}\right)\\
\text{s.t. (PC) } & U_{0}\left(C_{0}\right)\geq U_{0}\left(C_{0}^{a}(Y_{0})\right)\\
\text{(RP) } & \pi_{1}\left(C_{1}^{r}\left(C_{1}\right);C_{1}\right)\leq\kappa
\end{align*}


The first restriction is the same participation constraint as before.
The second is a restriction placed by the fact that marginal utilities
in periods 1 and 2 cannot be too far apart from period 1's perspective.
The new monopolist renegotiation-proof contract $C_{0}^{mp}$ will
keep consumption sufficiently imbalanced (in period 1's favor) so
that period 0's beliefs about the potential gains to period 1 from
renegotiating the contract are just smaller than the costs the bank
incurs from renegotiating.

A slight re-interpretation of FIGURE illustrates the properties of
a renegotiation-proof contract in the special case of a monopolist
with zero renegotiation cost ($\kappa=0$). In this special case the
only contract that satisfies the period 0 self's participation constraint
(interpreted here as indifference curve $FR$ where $F$ is no longer
the full-commitment contract since period 0 consumption must lie along
the first-order condition line $c_{2}=\beta^{(1/{\rho)}}c_{1}$ (which
line in the here for the CRRA case ).--JONATHAN to modify once figures
come in.

------

This yields an equilibrium which satisfies the borrower's participation
constraint. For the sophisticated hyperbolic discounter, since $c_{1}^{mp}$
and $c_{2}^{mp}$ deliver a (weakly) less smoothed profile than the
commitment contract, the bank has to give up some of its surplus to
have the client continue to participate. Bank profits must be lower
when the bank faces a new constraint on its surplus extraction problem:
the bank wishes it could promise to not renegotiate but it simply
cannot make such a promise credible without giving up some profits.
The problem here is not one of cheating or contract failure {[}AS
examined for example by Hansmann, Glaeser-Shleifer{]}, it is the possibility
of a legitimate renegotiation (a voluntary agreement to tear up the
old contract) between the consumer and the firm. 

{[}Point $R$ in FIGURE provides an illustration of the approximate
renegotiation-proof contract that a sophisticated consumer would agree
to with a $\ensuremath{\kappa=0}$ bank. It's only approximate because
the only renegotiation-proof con he bank faces zero renegotiation
cost they will offer to renegotiate.{]}--JONATHAN to modify once figures
come in.

The following lemma and proposition formalize the characteristics
of the renegotiation-proof contract.
\begin{lem}
(i) Profits are continuous over $\Omega$. (ii) On any line segment
that connects $Y_{0}^{\ast}$ to the boundary of $\Omega$, profits
are strictly increasing in distance from $Y_{0}^{\ast}$.\end{lem}
\begin{proof}
(i) For any amount allocated to periods $1$ and $2$, denoted $c_{f}$,
let $v\left(c_{f}\right)=\beta U_{1}\left(C_{1}\left(c_{f}\right)\right)$
be the maximized discounted utility from an allocation, $C_{1}\left(c_{f}\right)$,
that satisfies the renegotiation-proofness constraint. $v\left(c_{f}\right)$
is continuously rising in $c_{f}$. {[}prove rigorously: For any $C_{1}$,
let $\hat{C}_{1}\left(C_{1}\right)$ be the profit-maximizing renegotiated
contract. Since utility is continuous, for any $c_{f}^{\prime}$ that
approaches a given $c_{f}$, $\hat{C}_{1}\left(C_{1}\left(c_{f}^{\prime}\right)\right)$
must approach $\hat{C}_{1}\left(C_{1}\left(c_{f}\right)\right)$,
so $C_{1}\left(c_{f}^{\prime}\right)$ must approach $C_{1}\left(c_{f}\right)$.{]}
In period $0$, the bank must find the smallest $c_{1}+c_{f}$ such
that $u\left(c_{1}\right)+v\left(c_{f}\right)=U_{0}\left(Y_{0}\right)$.
Since $u$ and $v$ are continuously rising, $U_{0}$ is continuous
in $Y_{0}$, and profits are continuous in $C_{0}$ ($\bar{y}-\left(c_{1}+c_{f}\right)$),
profits must be continuous in $Y_{0}$. (ii) Consider two points,
$Y^{1}$ and $Y^{2}$, such that they lie on a line segment connecting
$Y_{0}^{\ast}$ to the boundary of $\Omega$, and $Y^{2}$ is closer
to $Y_{0}^{\ast}$ than is $Y^{2}$. Then, $U_{0}\left(Y^{1}\right)>U_{0}\left(Y^{2}\right)$.
So the contract that is optimal at $Y^{1}$ must satisfy both constraints
at $Y^{2}$. Furthermore, at $Y^{2}$, the participation constraint
will be slack, so the bank could raise profits by lowering $c_{0}$.
Therefore, $\pi\left(C_{0}^{mp};Y^{1}\right)<\pi\left(C_{0}^{mp};Y^{2}\right)$.\end{proof}
\begin{prop}
If $\kappa$ is sufficiently small: (i) Profits are strictly lower
than under the full-commitment contract. (ii) There is a simple closed
curve containing $Y_{0}^{\ast}$ such that a contract is offered if
and only if $Y_{0}$ lies outside the curve.\end{prop}
\begin{proof}
(i) Suppose $\kappa$ is sufficiently small that the renegotiation-proofness
constraint binds at a given $Y_{0}$. Then, profits must be strictly
lower than when the constraint does not exist. (ii) Firs t, since
$u^{\prime}\left(0\right)=\infty$, the bank can earn positive profits
at the boundary of $\Omega$ for any $\kappa$, however small. Suppose
$\kappa$ is sufficiently small that the renegotiation-proofness constraint
binds at $Y_{0}^{\ast}$. Then, profits at $Y_{0}^{\ast}$ are less
than zero. Since profits are decreasing along any line segment from
$Y_{0}^{\ast}$ to the boundary, and since profits are continuous
over $\Omega$, there must be a closed curve containing $Y_{0}^{\ast}$
along which profits are $0$. Therefore, the bank will not offer a
contract at any point inside the curve.
\end{proof}
The intuition behind this result if the following: if the initial
income/consumption stream is sufficiently close to the optimal, the
bank's contract, by opening up the possibility of renegotiation, cannot
do better. As a result, the bank will be unable to offer renegotiation-proof
contracts to those individuals whose autarky consumption streams are
sufficiently close to optimal.
\end{document}
